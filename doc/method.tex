\documentclass[]{article}
\usepackage[T1]{fontenc}
\usepackage{amsmath}
\usepackage{amsfonts}
\usepackage{amssymb}
\usepackage{mathrsfs}

% commands
\renewcommand{\xi}{\greekxi}
\renewcommand{\vec}{}
\newcommand{\Q}{\vec{Q}}
\newcommand{\q}{\vec{q}}
\newcommand{\xim}{x_{i-1}}
\renewcommand{\xi}{x_{i}}
\newcommand{\ximh}{x_{i-1/2}}
\newcommand{\xiph}{x_{i+1/2}}
\newcommand{\qimh}{q_{i-1/2}}
\newcommand{\qiph}{q_{i+1/2}}
\newcommand{\barkappa}{\bar{\kappa}}
\newcommand{\bareta}{\bar{\eta}}
\newcommand{\A}{\mathcal{A}}
\newcommand{\Z}{\mathcal{Z}}
\begin{document}

\noindent{\bf Objective}: produce a numerical scheme for wave propagation in spacetime-varying media.
\newline

In one spatial dimension wave propagation is described by\footnote{\noindent Note that in the problems we are interested in the flux $f$ does not depend explicitly on \emph{time}.},
\begin{equation}
	\barkappa(x,t)\cdot\q(x,t)_t + \vec{f}_x(\q,x) = \vec{\psi}(\q,x,t) 
\end{equation}

The numerical scheme is given by,
\begin{equation}
	\frac{\partial Q_i}{\partial t} = \frac{1}{K_i \Delta x}\left(  A^{+}\Delta\qimh + \A^{-}\Delta\qiph + \A\Delta\q_i \right)
\end{equation}
where
\begin{subequations}
	\begin{alignat}{1}
		\A^{-}\Delta q  &= \sum_{p,s^p\leq 0} \Z^p,\\
		\A^{+}\Delta q &= \sum_{p,s^p\geq 0} \Z^p,
	\end{alignat}
\end{subequations}
where $\Z$ is commonly known as the $f$--wave.

And the total fluctuation,
\begin{equation}
	f(\qiph^L) - f(\qimh^R) - \Delta x \psi(q_l,q_r)	
\end{equation}

We use $f$--wave Riemann solvers to decompose the flux difference in waves,
\begin{equation}
	f(q_r) -f(q_l) = \sum_p \beta^p r^p = \sum_p \Z^p,
\end{equation}

\section{Riemann solver}
For Maxwell's equation in 1D we have:
\begin{equation}
	\q = \left(\begin{array}{c} q_e \\ q_h \end{array}\right), \qquad \vec{f} = \left(\begin{array}{c} \frac{q_h}{\eta_e^0} \\ \frac{q_h}{\eta_h^0} \end{array}\right)
\end{equation}
with,
\begin{equation}
	\barkappa = \left({\begin{array}{cc} \eta_e(x,t) & 0 \\ 0 & \eta_h(x,t) \end{array}}\right)
\end{equation}
and, in the case of linear materials $\vec{\psi}=\partial_t\barkappa\cdot\q$.

The the flux Jacobian, $f_q$, becomes:
\begin{equation}\label{eq:appendix.riemann.1d.defqf}
	f_q = A =\left(\begin{array}{cc} 0 & 1/\eta_e^0 \\ 1/\eta_h^0 & 0 \end{array}\right).
\end{equation}
With resulting eigenvectors, $r^p$.and eigenvalues, $s^p$:
\begin{subequations}\label{eq:appendix.riemann.1d.r12}
	\begin{alignat}{3}
		r^1&=\left(\begin{array}{c} -Z \\ 1\end{array}\right), &\qquad s^1 &= -c(x,t)\label{eq:emlinear.1d.r1},\\
		r^2&=\left(\begin{array}{c}+Z \\ 1\end{array}\right), &\qquad s^2 &=  c(x,t)\label{eq:emlinear.1d.r2},
	\end{alignat}
\end{subequations}
where $Z$ is the impedance,
\begin{equation}\label{eq:emlinear.1d.z}
	Z=\sqrt{\frac{\eta_e^0}{\eta_h^0}}.
\end{equation}
and eigenvalues, $s^{1,2}$:
\begin{equation}\label{eq:appendix.riemann.1d.vpm}
	s^{1,2}=\pm c(x,t) = \frac{\pm 1}{\sqrt{\eta_e^0\eta_h^0}}.
\end{equation}
For the Riemann problem between the cells $i-1$ and $i$, the $f$--waves, $\Z^{1,2}$, are:
\begin{equation}\label{eq:appendix.riemann.1d.defw}
	\Z^1=\beta^1r^1_{i-1}, \qquad \Z^2=\beta^2r^2_{i}.
\end{equation}

Let $R=[r^1\:r^2]$, then we can determine $\beta^i$ by solving the system:
\begin{equation}\label{eq:appendix.riemann.1d.defbeta}
	R\beta = f_i(Q_i)-f_{i.1}(Q_{i-1}) = \Delta f.
\end{equation}
To obtain
\begin{subequations}\label{eq:appendix.riemann.1d.beta}
\begin{alignat}{1}
	\beta^1&=\frac{-\Delta f^1+\Delta f^2Z_i}{Z_i + Z_{i-1}}\label{eq:appendix.riemann.1d.beta1},\\
	\beta^2&=\frac{\Delta f^1+\Delta f^2Z_{i-1}}{Z_i + Z_{i-1}}\label{eq:appendix.riemann.1d.beta2}.
\end{alignat}
\end{subequations}

\section{Algorithm}
A succinct description of the numerical algorithm is,
\begin{enumerate}
	\item Set the initial cell averages of $\bareta_l(x_i,t=0)$, $\bareta_l(x_i,t=0)_t$ and $\vec{q}(x_i,t=0)\equiv \vec{Q}_i$
	\item Reconstruction
	\begin{enumerate}
		\item Solve the Riemann problem at $\ximh$ using the cell averages $\Q_{i-1},\Q_i$
		\item Using WENO compute reconstructed piecewise function
		\item Get states $\qimh^R,\qiph^L$
	\end{enumerate}
	\item Compute the fluctuations, $\A^{+}\Delta\qimh,\A^{-}\Delta\qiph$. We do this by solving the Riemann problem with initial states $(\qimh^L,\qiph^R)$. \emph{We do not incorporate the effect of the source on this step, rather we will subtract it in the total fluctuation}
	\item Calculate the total fluctuation $\A\Delta\q_i$, use states $\qiph^L,\qimh^R$
	\item Add the net effect of all waves going to the right, left and the total fluctuation
	\item Integrate in time and get $\Q^{n+1}$
\end{enumerate}


\end{document}